%%%%%%%%%%%%%%%%%%%%%%%%%%%%%%%%%%%
% Teacher Workload
%%%%%%%%%%%%%%%%%%%%%%%%%%%%%%%%%%%
\section{Teacher Workload}
\label{sec:teacher_workload}
One topic that has been debated much in Danish news is the problem about the amount of students in each class. The essence of the problem is that teachers no longer feel like they can tend to the needs of all their students and they are not getting any extra supplies or help to deal with the problem. The class sizes negatively affect the students' ability to learn and the overall well-being of everyone in the classroom\cite{Romme-Mlby2018GymnasierGymnasieskolen}. There have been made laws in the danish education system to prevent the overpopulation of classes and secure a high level of education for the students, but unfortunately these rules are not respected by all gymnasiums. An example of one of the laws against the overpopulation of classes is a flexible cap of 28 students per class instated by the danish government to make sure the gymnasiums cannot just keep stuffing more students in their classrooms. The cap being flexible means that some classes are allowed to have more students, as long as the average number students pr. class for the whole gymnasium is under 28. The reasoning behind the flexible cap is that it gives students the chance to enter the field of study they prefer. It can be seen in a study made in 2014 by \newacr{Gymnasieskolens Lærerforening}{GL} that the class cap is being broken by quite a large amount of gymnasium classes in Denmark. In the study 1758 danish high school classrooms were inspected, from these 1758 classrooms 728 classes had a population over 28 students which corresponds to 41.4\% of all classes inspected being over the class cap of 28 students per classroom \cite{KlassestrrelsersGymnasiet}. These larger class sizes bring with them a bunch of negative adverse effects.

\begin{itemize} 
    \item \textbf{Less individualized attention} is one of the effects of large class sizes. The the teacher will have a harder time identifying the needs and skills of each individual student. This in turn will lead to some students getting behind in their studies and not getting the knowledge they otherwise would \cite{Gibori2012TheStudents}.
    \item \textbf{A disruptive learning environment} is one of the other negative effects of increased class sizes. More students being stuffed into classrooms meant for fewer people will lead to poor concentration due to reduced air quality \cite{Gibori2012TheStudents}.
    \item \textbf{Less participation} will also be a result of overpopulated classes, as there will be more students, but still the same amount of time for students to participate in class discussions\cite{Gibori2012TheStudents}. 
\end{itemize}

\noindent
These adverse effects are to be avoided, as they clearly negatively affect the quality of education the students receive. These effects are also backed by another study in which 2300 teachers participated \cite{KlassestrrelsersGymnasiet}. The teachers who participated are representative of different gymnasium educations, geographical locations, gymnasium size, gender, subject and age. In this study 97\% of teachers say that reducing the number of students in the classes will increase the possibility of lifting the students academically. This clearly shows that the average teacher does not feel like they have the time and resources they need to lift their students as high as possible academically, which can also be seen through some of the interviewed teachers saying that the classes cap of 28 is more than enough.
\newline\indent
Furthermore, the study indicated that the larger class sizes harmed students that were both strong and weak academically. The obvious solution for the overpopulated classes would be to reduce the amount of students in each class. This endeavour has shown to be a costly one though. A study made by the independent danish thinktank Cevea shows that a reduction of 10\% to the current class cap would cost the danish government around 3.5 billion a year \cite{Cevea2013UtilsigtedeLsninger}. Another possibility for removing some of the stress the large class sizes put on the teachers is giving the teachers extra resources that give them a better overview of the academical capabilities of their students. This would as a bonus also would have the potential of reducing the preparation time for their lessons. This is in the interest of many teachers because the general opinion of teachers is that they do not have enough time for preparation of their lectures \cite{AntoniLund2017KortUdfordring}. 