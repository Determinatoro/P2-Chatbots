%%%%%%%%%%%%%%%%%%%%%%%%%%%%%%%%%%%
% Stakeholder Analysis
%%%%%%%%%%%%%%%%%%%%%%%%%%%%%%%%%%%
\section{Stakeholder Analysis} 
\label{sec:stakeholder}
In this section a stakeholder analysis has been made to encompass who and why certain stakeholders would be impacted if the project was to be carried out. To limit the stakeholder analysis, the analysis will only focus on stakeholders which would be directly impacted.

%%%%%%%%%%%%%%%%%%%%%%%%%%%
% Teachers
%%%%%%%%%%%%%%%%%%%%%%%%%%%
\subsection*{Teachers}
\label{sec:Teachers}
In a large questionnaire from \newacr{Gymnasieskolernes lærerforening}{GL} with a total of 2347 responses from Danish Gymnasium teachers from stx\footnote{Almen studentereksamen / General high school diploma}, hhx\footnote{Højere handelseksamen / Higher Commercial Examination}, htx\footnote{Teknisk studentereksamen / Technical degree} and hf\footnote{Højere forberedelseseksamen / Higher Preparatory Examination}; it is shown that 97\% of the teachers agreed that fewer students per class ensure a better educational environment \cite{Rasmussen2018NyGymnasieskolen}. 
\newline\newline
\enquote{Almost all teachers agree that the number of students per class is decisive for the benefits the students get from the lessons. Everyone who has been in a class room knows that if there are too many students in the class, then it becomes difficult to build a good relationship and create a dialogue where all students are speaking and where everyone gets qualified feedback.} Says Annette Nordstrøm in the questionnaire from GL \cite{Rasmussen2018NyGymnasieskolen}.
\newline\newline
As immersed in \textnameref{sec:teacher_workload}, it can be difficult for teachers to teach classes with many students. The large size classes negatively affects students ability to learn and the teachers ability to build a relationship with students.

%%%%%%%%%%%%%%%%%%%%%%%%%%%
% Students
%%%%%%%%%%%%%%%%%%%%%%%%%%%
\subsection*{Students}
Technology can be a great resource for students to enable themselves, both in the classroom but also at home where they can not ask other students or their teacher for help, but instead rely on their parents. Less privileged students who are not able to get help from their parents or siblings will complete their homework to a low standard or in the worst case skip it. Especially students who struggle will in a higher degree be able to learn subjects at their own pace; traditional classrooms tends to make it difficult to do so, this is better described in section \textnameref{sub:using_online_and_virtual_tools}.
\newline\newline
”The class size means everything. In a large class - with over 28 students - each student gets less help dealing with both academic, social and personal challenges / issues.” A teacher says in the questionnaire from GL \cite{Rasmussen2018NyGymnasieskolen}.
\newline\newline
Especially in large classrooms students struggle to get attention, this is a product of the teacher / student ratio where it will be more difficult for teachers to focus on each individual student and therefore also their strengths and weaknesses. This can have severe consequences for both the teachers and the students. This is amplified in \textnameref{sec:teacher_workload} which describes the consequences it can have for both the teachers and students if the teacher do not have enough time for each student.  

%%%%%%%%%%%%%%%%%%%%%%%%%%%
% Gymnasiums
%%%%%%%%%%%%%%%%%%%%%%%%%%%
\subsection*{Gymnasiums}
To ensure a high educational standard today's funding of Danish gymnasiums depends on the taximeter system. The taximeter system pushes gymnasiums to prioritize the quality of the education as well as educational-political goals set by the government. If gymnasiums produce good results and quality education, they will be granted more money. Gymnasiums also receive money depending on how many students they take and educate \cite{RegeringenUndervisningsministeriet}.
\newline\newline
Furthermore, you would also expect that providing a tool to help students learn math would surely result in higher average grades. This would then make the concerned gymnasiums more attractive for new students. This is because the grades a student receives at gymnasium affect which universities and courses they can attend. Universities use grade boundaries to filter which students they accept into their courses \cite{Tjek2}. If a student wishes to study a specific course, higher grades make them more likely to be accepted.



\begin{comment}
You can also see a connection between the gymnasium with the lowest grades in average and the gymnasiums which received the lowest amount of first priority applications. In 2017 the two gymnasiums in the danish capital region with the lowest amount applications was also the two gymnasiums to have the lowest exam grades \cite{FischerThomsenHerLorry}\cite{HecklenAlexanderKielgastFindDR}.

When new-coming students decide on what gymnasium to choose, they often go to influencers such as parents, teachers and education supervisors. A negative image of any school institution means that they are in risk of being opt-out to any alternative schools with a better image. The schools with a better image would 
\end{comment}



\begin{comment}


\section{Geometry}
Geometry on Mathematical C level deals figures in a 2- or 3 dimensional and generally only comes around 3 subtopics; units, area and volume and surface area. In this section, it will be covered what methods that will be used in the software product. 
%Intro
\subsection{Units}
\subsubsection{fff}

\[1m \cdot 1m = (1m)^2 = 1^2 m^2 = 1m^2\]

\[(1m)^2 = (100cm)^2 = 100^2cm^2=10.000cm^2\]

\subsubsection{Volumen}
\[1m^3=(10dm)^3=10^3dm^3=1000 liter\]
\subsection{Area}
\subsubsection{Rectangle}
\[A=l*b\] skal lige skrives på engelsk

\subsubsection{Right-angled triangle}
\[T=\frac{1}{2}\cdot a \cdot b\]
\[T=\frac{1}{2}\cdot h \cdot g\] 

\[\]

\subsubsection{Triangle}
\[T=\frac{1}{2}\cdot h \cdot g\] 

\subsubsection{Parallelogram}
\subsubsection{Trapez}
\subsubsection{Circle}

\subsection{Volume and Surface Area}

\subsubsection{Box}
\subsubsection{Cylinder}
\subsubsection{Prism}
\subsubsection{BALLS??}
\subsubsection{Cone}
\subsubsection{Pyramid}

% What
\section{Equations}
An equation is an expression which contains an equal sign. Typically when talking about equations, there is an unknown factor, most often referred to as \(x\). To solve an equation, the left and right side needs to be equal to each other. To figure out how the left and right side are equal to each other, there should be made calculations to figure out what \(x\) is.
%Why
\subsection{Solving Equations}
When solving equations, some methods are often applied to make it easier. One of these methods are called "equivalent equations". This methods aims to transform the equation while keeping in mind that the equations solution should be able to be applied on the original equation. 
\newline
Here is an example of how the equivalent equation method.
\[ 5x+30 = 60 \]
\[ 5x+30-30 = 60-30 \]
\[ \frac{5x}{5} = \frac{30}{5} \]
\[ x = 6 \]
\subsection{Solving Higher Equations}

\subsection*{Discriminant}

A discriminant is very exciting in some ways we do not know of yet.
\begin{align*}
    \text{No solutions - Discriminant < 0}    \\ 
    \text{One solution - Discriminant = 0}    \\
    \text{Two solutions - Discriminant > 0}       
\end{align*}




%*** What ***
%*** How ***




\end{comment}
























\begin{comment}

%%%%%%%%%%%%%%%%%%%%%%%%%%%
% Substitutes
%%%%%%%%%%%%%%%%%%%%%%%%%%%
\subsection*{Substitutes}
Substitutes are employees who are only hired for a temporary amount of time, mostly only one year at the time, even though shorter employment periods are possible too.  For the most part substitutes fill in for teachers who are not vacant for a prolonged period of time, if a teacher is sick or if there is a lack of teachers\cite{Vikar}. 
\newline\newline
Filling in for other teachers can mean that substitutes do not always have a chance to prepare themselves properly within lectures. This can cause issues if the substitute are not able to answer their questions to a high enough standard.
\newline\newline
Substitutes filling in for teachers for a prolonged amount of time means that the substitute will have to learn and identify their student's strong and weak sides, and build a relationship with the students before they can give them proper feedback \cite{Vikar}.

%Hvis man sammenligner med karaktererne kan man se en sammenhæng. De 2 gymnasier med de laveste karakterer har de færreste ansøgninger med førsteprioritet. Det er sværere at teste i andre regioner end i hovedstaden da man her ikke har så mange forskellige gymnasier at vælge imellem.

% https://www.tv2lorry.dk/artikel/her-er-regionens-mest-populaere-gymnasier
% https://www.dr.dk/nyheder/indland/find-dit-eget-gymnasium-saa-stor-er-forskellen-paa-aarskarakterer-og-eksamen

%How to maintain a good image.  
\end{comment}
