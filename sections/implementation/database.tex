%%%%%%%%%%%%%%%%%%%%%%%%%%%%%%%%%%%
% Database
%%%%%%%%%%%%%%%%%%%%%%%%%%%%%%%%%%%
\clearpage
\section{Database}

\missing

\Lstlistingref{lst:create_table_materials}

\begin{lstlisting}[style=SQLstyle, label=lst:create_table_materials, caption={This is the SQL code used for creating the Materials table}]
CREATE TABLE Materials (
	Id int PRIMARY KEY NOT NULL IDENTITY(1,1),

	TermId int,
	TopicId int,
	[Source] varchar(max) NOT NULL,
	ShowOrderId int NOT NULL,

	FOREIGN KEY (TopicId) REFERENCES [Topics](Id),
	FOREIGN KEY (TermId) REFERENCES [Terms](Id),

	CONSTRAINT CK_Materials_TermId_TopicId_only_of_them_not_null 
	CHECK ((TermId IS NOT NULL AND TopicId IS NULL) OR (TermId IS NULL AND TopicId IS NOT NULL)),
	CONSTRAINT UQ_Materials_TermId_ShowOrderId UNIQUE (TermId,ShowOrderId),
	CONSTRAINT UQ_Materials_TopicId_ShowOrderId UNIQUE (TopicId,ShowOrderId)
)
\end{lstlisting}

\clearpage
\begin{lstlisting}[style=SQLstyle]
CREATE TRIGGER [dbo].[SetShowOrderMaterials]
 ON [dbo].[Materials]
 INSTEAD OF INSERT
AS
BEGIN
  SET NOCOUNT ON;
  
  DECLARE @TermId int, @TopicId int, @ShowOrderId int, @Count int
          
  SELECT @TermId = TermId, @TopicId = TopicId FROM inserted
  
  IF (@TermId IS NOT NULL)
  BEGIN
	IF NOT EXISTS (SELECT 1 FROM Materials WHERE TermId = @TermId)
	BEGIN
		SET @ShowOrderId = 1
	END
	ELSE
	BEGIN
		SELECT @Count = COUNT(TermId) FROM Materials WHERE TermId = @TermId 
		SET @Count = @Count + 1

		SET @ShowOrderId = @Count
	END

	INSERT dbo.Materials (TermId, [Source], ShowOrderId)
		SELECT TermId, [Source], @ShowOrderId FROM inserted 
  END
  ELSE
  BEGIN
	IF NOT EXISTS (SELECT 1 FROM Materials WHERE TopicId = @TopicId)
	BEGIN
		SET @ShowOrderId = 1
	END
	ELSE
	BEGIN
		SELECT @Count = COUNT(TopicId) FROM Materials WHERE TopicId = @TopicId
		SET @Count = @Count + 1

		SET @ShowOrderId = @Count
	END

	INSERT dbo.Materials (TopicId, [Source], ShowOrderId)
		SELECT TopicId, [Source], @ShowOrderId FROM inserted 
  END
END
\end{lstlisting}

