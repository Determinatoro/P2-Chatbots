Constraints:
The constrains are all the possible outcomes that are for the experiment we are doing. If we throw a dice and are interested in how many eyes it shows, the constraints
U = {1,2,3,4,5,6}
So, there are 6 different outcomes.
If instead we had tossed two dice, each outcome would be two numbers. For example (4,3), that would mean that the first cube showed a 4 and the other a 3. In this case, the constraints would consist of 36 different outcomes (the first cube can show 6 different values and for each of them the second cube can show 6 different values. In total, there are 6·6 = 36 different options)
U = {(1,1), (1,2), (1,3), (1,4), (1,5), (1,6), (2,1), ..., ( 6,4), (6,5), (6,6)}
If we had played poker (where you get 5 out of 52 cards), a poker hand (an outcome) could have been (K5, S3, HE, RJ, KK) -  5 of Clubs, 3 of Spade 3, Ace of Hearts, Jack of diamonds and King of Clubs. A total of 2,598,960 different poker hands would be available. 
We could also have had a bowl of 3 red and 1 blue ball. If we draw a ball, the dropout would be
U = {Red, Blue}

Probability:
Each element in the constraints is associated with a probability. One describes the probability with a small p.
In the case of one dice, the probabilities of each outcome are the same. There are 6 sides on the dice, so the probability of each outcome is 1/6
p (1)= p (2)= p (3)= P (4)= p (5)= p (6)=1/6≈0,1667
In the case of two dice, there are a total of 36 possible outcomes. They are all equally likely, so the probability is
1/36≈0,02778
for each outcome.
If all outcomes are equally likely, we call it a symmetrical probability field. The two examples above are symmetrical probability fields. 
The example of the bowl with 4 balls, where 3 is red and 1 is blue is not symmetrical, then
P (red)=3/4  = 0.75,P (blue)=1/4  = 0.25
If you put the probabilities of all the elements together, it must give 1 (equivalent to 100%).

Occurrence:
An Occurence, H, is a subset of the constraints. If one could look at the occurence in the experiment with one dice
H = {Number of eyes odd}
The numbers on the dice in the constraints that meet this are 1, 3 and 5.
We mark the likelihood of an occurrence occurring with a large P. One finds the probability of an event by putting all the probabilities of the individual elements of the event together.
P (H)= p (1)+ P (3)+ p (5)=1/6  +1/6  +1/6  =3/6  = 0.5
If it is a symmetrical probability field, the probability of an event is
P (H)=  (Number of beneficial outcomes)/(Number of possible outcomes)
Forexsample could an occurrence with to dice throw be
H = {sum of eyes is 5}
The favorable outcomes are (1.4), (2.3), (3.2) and (4.1). That means that there are 4 favorable outcomes. We insert in the formula:
P (H)=4/36≈0,1111

Complementary event:
Sometimes it is easier to figure out the probability of an event not happening.
If our event is called H, then we mean the event that H does not occur with
H ̅
We call it, the complementary event. Either H is happening or it’s not. Therefore, the sum of the probabilities must be 1 (100\%)
P (H) + P (\={H}) = 1
P (H) = 1-P (\={H})

We hit with three dice and want to find the probability that we get at least one six. So our event is
H = {at least 1 six}
However, calculating how many beneficial outcomes are for this event is not quite easy.
The complementary event must be:
\={H}= {no six}
It is somewhat easier to calculate the likelihood of this event occurring. In fact, when there are no sixes, there are five favorable outcomes on the first dice (1, 2, 3, 4 or 5). For each of them there are 5 favorable outcomes on the next dice, and for each of them there are 5 favorable outcomes on the third. So the probability must be
P(\={H})=  ( Number of beneficial outcomes)/(Number of possible outcomes)  =(5·5·5)/(6·6·6)  =125/216≈0.579
By subtracting this probability from 1, we get the probability of H.
P (H) = 1-P (\={H}) = 1-0.579 = 0.421
So there is 42.1\% probability of getting at least one six, if you have three strokes.
