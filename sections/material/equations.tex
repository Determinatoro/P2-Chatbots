\section{Equations}
An equation is an expression which contains an equal sign. Typically when talking about equations, there is an unknown factor, most often referred to as \(x\). To solve an equation, the left and right side needs to be equal to each other. To figure out how the left and right side are equal to each other, there should be made calculations to figure out what \(x\) is.
\subsection{Solving Equations}
When solving equations, some methods are often applied to make it easier. One of these methods are called "equivalent equations". This methods aims to transform the equation while keeping in mind that the equations solution should be able to be applied on the original equation. 
\newline
Here is an example of how the equivalent equation method.
\[ 5x+30 = 60 \]
\[ 5x+30-30 = 60-30 \]
\[ \frac{5x}{5} = \frac{30}{5} \]
\[ x = 6 \]
\subsection{Solving Higher Equations}

\subsection*{Discriminant}

A discriminant is very exciting in some ways we do not know of yet.
\begin{align*}
    \text{No solutions - Discriminant < 0}    \\
    \text{One solution - Discriminant = 0}    \\
    \text{Two solutions - Discriminant > 0}       
\end{align*}




%*** What ***
%*** How ***