%  A simple AAU report template.
%  2015-05-08 v. 1.2.0
%  Copyright 2010-2015 by Jesper Kjær Nielsen <jkn@es.aau.dk>
%
%  This is free software: you can redistribute it and/or modify
%  it under the terms of the GNU General Public License as published by
%  the Free Software Foundation, either version 3 of the License, or
%  (at your option) any later version.
%
%  This is distributed in the hope that it will be useful,
%  but WITHOUT ANY WARRANTY; without even the implied warranty of
%  MERCHANTABILITY or FITNESS FOR A PARTICULAR PURPOSE.  See the
%  GNU General Public License for more details.
%
%  You can find the GNU General Public License at <http://www.gnu.org/licenses/>.
%
\documentclass[11pt,a4paper,openright, hidelinks]{report}
%%%%%%%%%%%%%%%%%%%%%%%%%%%%%%%%%%%%%%%%%%%%%%%%
% Language, Encoding, Comments and Fonts
% http://en.wikibooks.org/wiki/LaTeX/Internationalization
%%%%%%%%%%%%%%%%%%%%%%%%%%%%%%%%%%%%%%%%%%%%%%%%
% Select encoding of your inputs. Depends on
% your operating system and its default input
% encoding. Typically, you should use
%   Linux  : utf8 (most modern Linux distributions)
%            latin1 
%   Windows: ansinew
%            latin1 (works in most cases)
%   Mac    : applemac
% Notice that you can manually change the input
% encoding of your files by selecting "save as"
% an select the desired input encoding. 
\usepackage[utf8]{inputenc}
% Make latex understand and use the typographic
% rules of the language used in the document.
\usepackage[danish,english]{babel}
% Use the palatino font
\usepackage[sc]{mathpazo}
% Palatino needs more leading (space between lines)
\linespread{1.05}         
% Choose the font encoding
\usepackage[T1]{fontenc}
% Out comment text
\usepackage{comment}
% Quote text
\usepackage{csquotes}
\usepackage{float}
\usepackage{color,soul}
\addto\captionsdanish{%
  \renewcommand\appendixname{Bilag}
  \renewcommand\appendixpagename{Bilag}
  \renewcommand\appendixtocname{Bilag}
}

% Please add the following required packages to your document preamble:
\usepackage{booktabs}
\usepackage{longtable}
\usepackage[table,xcdraw,dvipsnames]{xcolor}
\usepackage{diagbox}
% If you use beamer only pass "xcolor=table" option, i.e. \documentclass[xcolor=table]{beamer}

%%%%%%%%%%%%%%%%%%%%%%%%%%%%%%%%%%%%%%%%%%%%%%%%
% Graphics and Tables
% http://en.wikibooks.org/wiki/LaTeX/Importing_Graphics
% http://en.wikibooks.org/wiki/LaTeX/Tables
% http://en.wikibooks.org/wiki/LaTeX/Colors
%%%%%%%%%%%%%%%%%%%%%%%%%%%%%%%%%%%%%%%%%%%%%%%%
% load a colour package
\usepackage{xcolor}
\definecolor{aaublue}{RGB}{33,26,82}% dark blue
% The standard graphics inclusion package
\usepackage{graphicx}
% Set up how figure and table captions are displayed
\usepackage{caption}
\usepackage{subcaption}
\captionsetup{%
  font=footnotesize,% set font size to footnotesize
  labelfont=bf % bold label (e.g., Figure 3.2) font
}
% Make the standard latex tables look so much better
\usepackage{array,booktabs}
% Enable the use of frames around, e.g., theorems
% The framed package is used in the example environment
\usepackage{framed}
% Make a table
\usepackage{multirow}
\usepackage{tabularx}
\usepackage{tabu}
\usepackage{wrapfig}
\usepackage[export]{adjustbox}
\usepackage{lipsum}
\usepackage{array}
\usepackage{rotating}
% List styles
\usepackage{lmodern}
\usepackage{enumitem}
\usepackage[export]{adjustbox}

%%%%%%%%%%%%%%%%%%%%%%%%%%%%%%%%%%%%%%%%%%%%%%%%
% Mathematics
% http://en.wikibooks.org/wiki/LaTeX/Mathematics
%%%%%%%%%%%%%%%%%%%%%%%%%%%%%%%%%%%%%%%%%%%%%%%%
% Defines new environments such as equation,
% align and split 
\usepackage{amsmath}
% Adds new math symbols
\usepackage{amssymb}
% Use theorems in your document
% The ntheorem package is also used for the example environment
% When using thmmarks, amsmath must be an option as well. Otherwise \eqref doesn't work anymore.
\usepackage[framed,amsmath,thmmarks]{ntheorem}

%%%%%%%%%%%%%%%%%%%%%%%%%%%%%%%%%%%%%%%%%%%%%%%%
% Page Layout
% http://en.wikibooks.org/wiki/LaTeX/Page_Layout
%%%%%%%%%%%%%%%%%%%%%%%%%%%%%%%%%%%%%%%%%%%%%%%%
% Change margins, papersize, etc of the document
\usepackage[
  inner=28mm,% left margin on an odd page
  outer=28mm % right margin on an odd page
  ]{geometry}
% Modify how \chapter, \section, etc. look
% The titlesec package is very configureable
\usepackage[T1]{fontenc}
\usepackage{titlesec, blindtext, color}
\newcommand{\hsp}{\hspace{5pt}}
%\titleformat{\thechapter}[display]{\Huge\bfseries}{\thechapter\hsp}{10pt}{\Huge}
\titleformat{\chapter}[hang]{\normalfont\Huge\bfseries}{\thechapter}{15pt}{\normalfont\Huge\bfseries}
\titleformat*{\section}{\normalfont\Large\bfseries}
\titleformat*{\subsection}{\normalfont\large\bfseries}
\titleformat*{\subsubsection}{\normalfont\normalsize\bfseries}
\titleformat*{\subparagraph}{\normalfont\normalsize\bfseries}
\titlespacing*{\chapter}{0pt}{0pt}{20pt}

% Clear empty pages between chapters
\let\origdoublepage\cleardoublepage
\newcommand{\clearemptydoublepage}{%
  \clearpage
  {\pagestyle{empty}\origdoublepage}%
}
\let\cleardoublepage\clearemptydoublepage

% Change the headers and footers
\usepackage{fancyhdr}
\pagestyle{fancy}
\fancyhf{} %delete everything
%\renewcommand{\headrulewidth}{0pt} %remove the horizontal line in the header
%\fancyhead[RE]{\small\nouppercase\leftmark} %even page - chapter title
%\fancyhead[LO]{\small\nouppercase\rightmark} %uneven page - section title
%\rhead{\chaptermark}
\renewcommand{\chaptermark}[1]{\markboth{#1}{}}
\fancyhead[LE,RO]{\chaptermark{TEST}} %page number on all pages
\cfoot{\thepage}
% Do not stretch the content of a page. Instead,
% insert white space at the bottom of the page
% Enable arithmetics with length. Useful when
% typesetting the layout.
\usepackage{calc}
\usepackage{pagecolor}
\usepackage{tikz}

%%%%%%%%%%%%%%%%%%%%%%%%%%%%%%%%%%%%%%%%%%%%%%%%
% Bibliography
% http://en.wikibooks.org/wiki/LaTeX/Bibliography_Management
%%%%%%%%%%%%%%%%%%%%%%%%%%%%%%%%%%%%%%%%%%%%%%%%
\DeclareUnicodeCharacter{5C5}{1234}
\usepackage[backend=bibtex,style=nature,bibencoding=utf8]{biblatex}
\addbibresource{bib/p2mendeley_jakob}
\setcounter{biburllcpenalty}{7000}
\setcounter{biburlucpenalty}{8000}


%%%%%%%%%%%%%%%%%%%%%%%%%%%%%%%%%%%%%%%%%%%%%%%%
% Misc
%%%%%%%%%%%%%%%%%%%%%%%%%%%%%%%%%%%%%%%%%%%%%%%%
% Add bibliography and index to the table of
% contents
\usepackage[nottoc]{tocbibind}
% Add the command \pageref{LastPage} which refers to the
% page number of the last page
\usepackage{lastpage}
% Add todo notes in the margin of the document
\usepackage[
%  disable, %turn off todonotes
  colorinlistoftodos, %enable a coloured square in the list of todos
  textwidth=\marginparwidth, %set the width of the todonotes
  textsize=scriptsize, %size of the text in the todonotes
  ]{todonotes}

%%%%%%%%%%%%%%%%%%%%%%%%%%%%%%%%%%%%%%%%%%%%%%%%
% Hyperlinks
% http://en.wikibooks.org/wiki/LaTeX/Hyperlinks
%%%%%%%%%%%%%%%%%%%%%%%%%%%%%%%%%%%%%%%%%%%%%%%%
% Enable hyperlinks and insert info into the pdf
% file. Hypperref should be loaded as one of the 
% last packages
\setlength{\headheight}{13.6pt}
\usepackage{hyperref}
\hypersetup{%
	pdfpagelabels=true,%
	plainpages=false,%
	pdfauthor={Author(s)},%
	pdftitle={Title},%
	pdfsubject={Subject},%
	bookmarksnumbered=true,%
	colorlinks=false,%
	citecolor=black,%
	filecolor=black,%
	linkcolor=black,% you should probably change this to black before printing
	urlcolor=black,%
	pdfstartview=FitH%
}

%%%%%%%%%%%%%%%%%%%%%%%%%%%%%%%%%%%%%%%%%%%%%%%%
% APPENDIX, GLOSSARY
%%%%%%%%%%%%%%%%%%%%%%%%%%%%%%%%%%%%%%%%%%%%%%%%

\usepackage[toc,page]{appendix}
\usepackage{csquotes}
\usepackage[toc]{glossaries}
\makeglossaries

%%%%%%%%%%%%%%%%%%%%%%%%%%%%%%%%%%%%%%%%%%%%%%%%
% LISTINGS
%%%%%%%%%%%%%%%%%%%%%%%%%%%%%%%%%%%%%%%%%%%%%%%%
\usepackage{listings}
\usepackage{tcolorbox}

%%%%%%%%%%%%%%%%%%%%%%%%%%%%%%%%%%%%%%%%%%%%%%%%
% COLORS
%%%%%%%%%%%%%%%%%%%%%%%%%%%%%%%%%%%%%%%%%%%%%%%%
\definecolor{codegreen}{RGB}{0,128,0}
\definecolor{codeblue}{RGB}{0,0,128}
\definecolor{codeblue2}{RGB}{0,0,255}
\definecolor{codemargenta}{RGB}{255,99,255}
\definecolor{codered}{RGB}{128,0,0}
\definecolor{lightgray}{rgb}{0.95, 0.95, 0.95}
\definecolor{backgray}{RGB}{250,250,250}
\definecolor{backcode}{RGB}{255,255,255}
\definecolor{aaucolor}{RGB}{0,36,86}
\definecolor{titlepagecolor}{RGB}{240,240,240}
\definecolor{musthavecolor}{RGB}{226,68,55}
\definecolor{shouldhavecolor}{RGB}{255,229,76} 
\definecolor{nicetohavecolor}{RGB}{53,141,19}

\definecolor{xmlbrown}{RGB}{163,21,21}
\definecolor{xmlred}{RGB}{255,39,20}
\definecolor{xmlgreen}{RGB}{34,128,0}
\definecolor{xmlblue}{RGB}{0,65,243}

%%%%%%%%%%%%%%%%%%%%%%%%%%%%%%%%%%%%%%%%%%%%%%%%
% CODE
%%%%%%%%%%%%%%%%%%%%%%%%%%%%%%%%%%%%%%%%%%%%%%%%
\usepackage{amsmath}
\usepackage{amsfonts}
\usepackage{algorithm}
\usepackage{algpseudocode}
\usepackage{algorithmicx}
\algrenewcommand\alglinenumber[1]{\tiny #1:}
\algnewcommand{\LeftComment}[1]{\Statex \(\triangleright\) #1}

\makeatletter
\renewcommand{\ALG@name}{Algorithm}
\renewcommand{\listalgorithmname}{List of \ALG@name s}
\makeatother
\makeatletter
\newenvironment{breakablealgorithm}
  {% \begin{breakablealgorithm}
   \begin{center}
     \refstepcounter{algorithm}% New algorithm
     \hrule height.8pt depth0pt \kern2pt% \@fs@pre for \@fs@ruled
     \renewcommand{\caption}[2][\relax]{% Make a new \caption
       {\raggedright\textbf{\ALG@name~\thealgorithm} ##2\par}%
       \ifx\relax##1\relax % #1 is \relax
         \addcontentsline{loa}{algorithm}{\protect\numberline{\thealgorithm}##2}%
       \else % #1 is not \relax
         \addcontentsline{loa}{algorithm}{\protect\numberline{\thealgorithm}##1}%
       \fi
       \kern2pt\hrule\kern2pt
     }
  }{% \end{breakablealgorithm}
     \kern2pt\hrule\relax% \@fs@post for \@fs@ruled
   \end{center}
  }
\makeatother
\algrenewcommand\textproc{\textsc}

\renewcommand{\lstlistingname}{Codesnippet}
% set the default code style
\lstdefinestyle{Cstyle}{
    language=c,
    basicstyle=\tiny\fontencoding{T1}\ttfamily,
    tabsize=4, 
    frame=tb,
    showstringspaces=false, 
    numbers=left, 
    commentstyle=\color{codegreen},
    keywordstyle=\color{codeblue},
    stringstyle=\color{codegreen},
    captionpos=b,
    xleftmargin=0.04\textwidth,
    backgroundcolor=\color{backgray}
}

\lstdefinestyle{SQLstyle}{
    language=SQL,
    showspaces=false,
    tabsize=4,
    frame=tb,
    numbers=left,
    captionpos=b,
    xleftmargin=0.04\textwidth,
    showtabs=false,
    breaklines=true,
    showstringspaces=false,
    breakatwhitespace=true,
    escapeinside={(*@}{@*)},
    commentstyle=\color{codegreen},
    keywordstyle=\color{codeblue2}\bfseries,
    stringstyle=\color{codered},
    basicstyle=\fontencoding{T1}\ttfamily\tiny,
    backgroundcolor=\color{backcode},
    morekeywords={if, begin, references, is},
    emph={count, max},
    emphstyle={\color{codemargenta}}
}

\lstdefinestyle{CSharpstyle}{
    language=[Sharp]C,
    showspaces=false,
    tabsize=4,
    frame=tb,
    numbers=left,
    captionpos=b,
    xleftmargin=0.04\textwidth,
    showtabs=false,
    breaklines=true,
    showstringspaces=false,
    breakatwhitespace=true,
    escapeinside={(*@}{@*)},
    commentstyle=\color{codegreen},
    keywordstyle=\color{codeblue2}\bfseries,
    stringstyle=\color{codered},
    basicstyle=\fontencoding{T1}\ttfamily\tiny,
    backgroundcolor=\color{backcode}
}

\lstdefinelanguage{XML}
{
  morestring=[s][\color{xmlblue}]{"}{"},
  morestring=[s][\color{black}]{>}{<},
  morecomment=[s]{<?}{?>},
  morecomment=[s][\color{xmlgreen}]{<!--}{-->},
  stringstyle=\color{black},
  identifierstyle=\color{xmlred},
  keywordstyle=\color{xmlbrown},
  morekeywords={CheckBox, Grid, Binding, StaticResource, RowDefinitions, RowDefinition, xmlns,xsi,noNamespaceSchemaLocation,type,id,x,y,source,target,version,tool,transRef,roleRef,objective,eventually}% list your attributes here
}



\definecolor{dkgreen}{rgb}{0,0.6,0}
\definecolor{gray}{rgb}{0.5,0.5,0.5}
\definecolor{mauve}{rgb}{0.58,0,0.82}
\definecolor{gray}{rgb}{0.4,0.4,0.4}
\definecolor{darkblue}{rgb}{0.0,0.0,0.6}
\definecolor{lightblue}{rgb}{0.0,0.0,0.9}
\definecolor{cyan}{rgb}{0.0,0.6,0.6}
\definecolor{darkred}{rgb}{0.6,0.0,0.0}


\lstdefinestyle{XMLstyle}{
  language=XML,
  showspaces=false,
  tabsize=4,
  frame=tb,
  numbers=left,
  captionpos=b,
  xleftmargin=0.04\textwidth,
  showtabs=false,
  breaklines=true,
  showstringspaces=false,
  breakatwhitespace=true,
  escapeinside={(*@}{@*)},
  stringstyle=\color{codered},
  basicstyle=\fontencoding{T1}\ttfamily\tiny,
  backgroundcolor=\color{backcode}
}






%%%%%%%%%%%%%%%%%%%%%%%%%%%%%%%%%%%%%%%%%%%%%%%%
% COMMANDS
%%%%%%%%%%%%%%%%%%%%%%%%%%%%%%%%%%%%%%%%%%%%%%%%
\usepackage{ifthen}

\newcommand{\newacr}[2]
{\emph{#1} (#2)}
\newcommand{\figureref}[1]
{figure \ref{#1}}
\newcommand{\Figureref}[1]
{Figure \ref{#1}}
\newcommand{\lstlistingref}[1]
{codesnippet \ref{#1}}
\newcommand{\Lstlistingref}[1]
{Codesnippet \ref{#1}}
\newcommand{\tableref}[1]
{table \ref{#1}}
\newcommand{\Tableref}[1]
{Table \ref{#1}}
\newcommand{\algoref}[1]
{algorithm~\ref{#1}}
\newcommand{\Algoref}[1]
{Algorithm~\ref{#1}}
\newcommand{\textnameref}[1]
{\ref{#1} \textit{\nameref{#1}}}
\newcommand{\mcite}
{\colorbox{Yellow}{\textbf{CITE}}}
\newcommand{\missing}
{\colorbox{Yellow}{\textbf{MISSING}}}
\newcommand{\intro}
{\colorbox{Yellow}{\textbf{INTRO}}}
\newcommand{\nicetohave}
{\colorbox{nicetohavecolor}{\color{white}\fontsize{8pt}{6pt}\fontfamily{phv}\selectfont NICE TO HAVE}}
\newcommand{\musthave}
{\colorbox{musthavecolor}{\color{white}\fontsize{8pt}{6pt}\fontfamily{phv}\selectfont MUST HAVE}}
\newcommand{\shouldhave}
{\colorbox{shouldhavecolor}{\color{black}\fontsize{8pt}{6pt}\fontfamily{phv}\selectfont SHOULD HAVE}}



\makeatletter
\renewcommand\paragraph{%
    \@startsection{paragraph}{4}{0mm}%
       {-\baselineskip}%
       {.5\baselineskip}%
       {\normalfont\small\bfseries}}
\makeatother

\usepackage{threeparttable}
\newcolumntype{L}[1]{>{\raggedright\arraybackslash}p{#1}}
\newcolumntype{C}[1]{>{\centering\arraybackslash}p{#1}}




